1.看到数学概念就想c++包装:
 看到一个点   想象成R_2空间中一点。-->
c++ 一个类,
template<>
class point{
data;
member function;
}

2.自然的对应
(数学概念固有的属性 点: coordinate[])
(数学概念的固有操作 点: 求模 double norm()
              求点的位置 double coordinate()
              +,-,×,/dot operator
判断是否相同 bool operator==())

3.写一个文档,包含class的设计
加上UML
(等忘记细节后)检查:
1. 层级关系
(上层简单调用下层函数,下层底层实现必要函数。)
2.每个类可用属性
3.单个类的复杂度
4.功能重用
5.可拓展性和复杂度的平衡
6.是否便于测试



4.对每一个模块函数证明正确性
1.明确契约(input (点,线, eps),output (是否在Yin集内部,外部,边上), precondition for input(线是jordan线), postcondition(在边上等价于到边的距离小于等于eps) )
2.设计算法 <-> 证明正确(algorithm包写出算法和证明)
3.足够细化
4.写下来


5.写程序
从底层开始,每一层进行调试后。再进行下一层。

6.调试